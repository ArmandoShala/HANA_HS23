%% Generated by Sphinx.
\def\sphinxdocclass{jupyterBook}
\documentclass[letterpaper,10pt,english]{jupyterBook}
\ifdefined\pdfpxdimen
   \let\sphinxpxdimen\pdfpxdimen\else\newdimen\sphinxpxdimen
\fi \sphinxpxdimen=.75bp\relax
%% turn off hyperref patch of \index as sphinx.xdy xindy module takes care of
%% suitable \hyperpage mark-up, working around hyperref-xindy incompatibility
\PassOptionsToPackage{hyperindex=false}{hyperref}

\PassOptionsToPackage{warn}{textcomp}

\catcode`^^^^00a0\active\protected\def^^^^00a0{\leavevmode\nobreak\ }
\usepackage{cmap}
\usepackage{fontspec}
\defaultfontfeatures[\rmfamily,\sffamily,\ttfamily]{}
\usepackage{amsmath,amssymb,amstext}
\usepackage{polyglossia}
\setmainlanguage{english}



\setmainfont{FreeSerif}[
  Extension      = .otf,
  UprightFont    = *,
  ItalicFont     = *Italic,
  BoldFont       = *Bold,
  BoldItalicFont = *BoldItalic
]
\setsansfont{FreeSans}[
  Extension      = .otf,
  UprightFont    = *,
  ItalicFont     = *Oblique,
  BoldFont       = *Bold,
  BoldItalicFont = *BoldOblique,
]
\setmonofont{FreeMono}[
  Extension      = .otf,
  UprightFont    = *,
  ItalicFont     = *Oblique,
  BoldFont       = *Bold,
  BoldItalicFont = *BoldOblique,
]


\usepackage[Bjarne]{fncychap}
\usepackage[,numfigreset=0,mathnumfig]{sphinx}

\fvset{fontsize=\small}
\usepackage{geometry}


% Include hyperref last.
\usepackage{hyperref}
% Fix anchor placement for figures with captions.
\usepackage{hypcap}% it must be loaded after hyperref.
% Set up styles of URL: it should be placed after hyperref.
\urlstyle{same}

\addto\captionsenglish{\renewcommand{\contentsname}{Software}}

\usepackage{sphinxmessages}



         \usepackage[Latin,Greek]{ucharclasses}
        \usepackage{unicode-math}
        % fixing title of the toc
        \addto\captionsenglish{\renewcommand{\contentsname}{Contents}}
        

\title{Höhere Analysis und Numerik}
\date{Aug 17, 2021}
\release{}
\author{Simon Stingelin}
\newcommand{\sphinxlogo}{\vbox{}}
\renewcommand{\releasename}{}
\makeindex
\begin{document}

\pagestyle{empty}
\sphinxmaketitle
\pagestyle{plain}
\sphinxtableofcontents
\pagestyle{normal}
\phantomsection\label{\detokenize{intro::doc}}


Das Skript zum Modul “Höhere Analysis und Numerik” ist ein interaktives Skript. Die Codes sind direkt ausführbar.

\begin{sphinxadmonition}{note}{Note:}
Das Skript zum Modul ist am entstehen.
\end{sphinxadmonition}

Der erste Teil des Skripts folgt dem Buch {[}{]} …


\part{Software}


\chapter{Python}
\label{\detokenize{Software/Python:python}}\label{\detokenize{Software/Python::doc}}
Für numerische Berechungen benutzen wir im Kurs Python. Eine gute Einführung in Python ist im interaktiven Jupyter\sphinxhyphen{}Book \sphinxhref{https://pythonnumericalmethods.berkeley.edu/notebooks/Index.html}{Python Programming And Numerical Methods: A Guide For Engineers And Scientists} zu finden.

Im wesentlichen werden wir fundamentale Funktionalität von Python benutzen. Für numerische Berechnungen werden wir auf NumPy und SciPy zurückgreifen. Die Visualisierung kann sehr matlab nahe mit Hilfe von Matplotlib umgesetzt werden.
\begin{itemize}
\item {} 
\sphinxhref{https://python.org}{Python}

\item {} 
\sphinxhref{https://numpy.org}{NumPy}

\item {} 
\sphinxhref{https://scipy.org}{SciPy}

\item {} 
\sphinxhref{https://matplotlib.org}{Matplotlib}

\end{itemize}

Falls Sie Python mit den entsprechenden Module noch nie benutzt haben, ist es sehr sinnvoll sich vorab mit dem wesentlichen auseinander zu setzen.


\chapter{NGSolve}
\label{\detokenize{Software/NGSolve:ngsolve}}\label{\detokenize{Software/NGSolve::doc}}
Im Rahmen der numerischen Methoden für partielle Differentialgleichungen werden wir die Methode der finiten Elemente kennen lernen. Wir werden für die Umsetzung auf eine C++ Bibliothek \sphinxhref{https://ngsolve.org}{NGSolve} zurückgreifen, welche eine umfangreiche Python\sphinxhyphen{}Schnittstelle zur Verfügung stellt.

Netgen/NGSolve ist eine hochleistungsfähige Multiphysik\sphinxhyphen{}Finite\sphinxhyphen{}Elemente\sphinxhyphen{}Software. Sie wird häufig zur Analyse von Modellen aus den Bereichen Festkörpermechanik, Strömungsmechanik und Elektromagnetik eingesetzt. Dank der flexiblen Python\sphinxhyphen{}Schnittstelle können neue physikalische Gleichungen und Lösungsalgorithmen leicht implementiert werden.

Das Ziel des Moduls besteht in der Vermittlung der mathematischen Grundlagen und insbesondere auch numerischen Anwendung derer auf konkrete Beispiele aus der Ingenieur Praxis.

\begin{sphinxadmonition}{note}{Note:}
Installieren Sie NGSolve auf ihrem Rechner. Die dazu notwendigen Anleitungen finden Sie auf (\sphinxurl{https://ngsolve.org}) unter dem Abschnitt INSTALLATION:
\begin{itemize}
\item {} 
\sphinxhref{https://ngsolve.org/downloads}{Downloads NGSolve}

\item {} 
\sphinxhref{https://docu.ngsolve.org/latest/install/usejupyter.html}{Using Jupyter notebook}

\end{itemize}
\end{sphinxadmonition}


\part{Teil 1}


\chapter{Funktionalanalysis}
\label{\detokenize{Funktionalanalysis/Funktionalanalysis:funktionalanalysis}}\label{\detokenize{Funktionalanalysis/Funktionalanalysis::doc}}
In diesem Kapitel wird das notwendige Rüstzeug für die Behandlung partieller Differentialgleichungen und numerischen Methoden bereit gestellt.


\section{Grundlegende Räume}
\label{\detokenize{Funktionalanalysis/Funktionalanalysis:grundlegende-raume}}

\subsection{Vektorräume}
\label{\detokenize{Funktionalanalysis/Funktionalanalysis:vektorraume}}
In den Modulen linearen Algebra, Analysis und Numerik wurden Vektorräume und Normen mit dem \(\mathbb{R}^n\) eingeführt und benutzt. Wir definieren hier der Vollständigkeit halber die Begriffe nochmals und erweitern die Anwendung auf allgemeinere Räume, insbesondere müssen diese nicht endlich dimensional sein.

Beginnen wir mit dem Begriff der Vektorräume. Um diesen definieren zu können, benötigen wir einen Zahlenkörper. In aller Regel benutzen wir die reellen Zahlen \(\mathbb{R}\). Wir legen mit dem folgenden Axiom fest, was die reellen Zahlen sind:

\begin{sphinxadmonition}{note}{Definition: reelle Zahlen}

Es existiere eine Menge \(\mathbb{R}\) mit folgenden Eigenschaften:
\begin{itemize}
\item {} 
Es existieren Operationen
\begin{equation*}
\begin{split}\begin{split}  + : \mathbb{R} \times \mathbb{R} \ & \to \ \mathbb{R}\quad\text{Addition}\\
  \cdot : \mathbb{R} \times \mathbb{R} \ & \to \ \mathbb{R}\quad\text{Multiplikation}\end{split}\end{split}
\end{equation*}
mit den Eigenschaften:
\begin{itemize}
\item {} 
Assoziativgesetze
\begin{equation*}
\begin{split}\begin{split}
    (a+b) + c & = a + (b+c)\quad \forall\ a,b,c \in \mathbb{R}\\
    (a\cdot b) \cdot c & = a \cdot (b\cdot c)\quad \forall\ a,b,c \in \mathbb{R}
    \end{split}\end{split}
\end{equation*}
\item {} 
Kommutativgesetze
\begin{equation*}
\begin{split}\begin{split}
    a + b & = b + a \quad \forall\ a,b \in \mathbb{R}\\
    a\cdot b & = b \cdot a\quad \forall\ a,b \in \mathbb{R}
    \end{split}\end{split}
\end{equation*}
\item {} 
Neutrale Elemente
\begin{equation*}
\begin{split}\begin{split}
    \exists 0: a + 0 = a\quad \forall a \in \mathbb{R}\\
    \exists 1: a \cdot 1 = a\quad \forall a \in \mathbb{R}
    \end{split}\end{split}
\end{equation*}
\item {} 
Inverse Elemente
\begin{equation*}
\begin{split}\begin{split}
    \forall a\in \mathbb{R}\ \exists a'\in\mathbb{R} : a + a' = 0\\
    \forall a\in \mathbb{R}\setminus \{ 0\}\ \exists \tilde{a}\in\mathbb{R} : a \cdot \tilde{a} = 1
    \end{split}\end{split}
\end{equation*}
Schreibweise: \(-a := a', a - a := a + (-a), \frac{1}{a} := \tilde{a}, \frac{a}{a} := a\cdot \frac{1}{a}\)

\item {} 
Distributivgesetz
\begin{equation*}
\begin{split}a\cdot (b+c) = a\cdot b + a\cdot c\end{split}
\end{equation*}
\end{itemize}

\item {} 
\(\mathbb{R}\) ist geordnet, d.h. es existiert eine Relation \(\le\) so, dass gilt
\begin{itemize}
\item {} 
\(\mathbb{R}\) ist totalgeordnet, d.h.
\begin{enumerate}
\sphinxsetlistlabels{\arabic}{enumi}{enumii}{}{.}%
\item {} 
\(\mathbb{R}\) ist teilgeordnet, d.h.
\begin{itemize}
\item {} 
Für alle \(x\in\mathbb{R}\) gilt: \(x \le x\).

\item {} 
Ist \(x \le y\) und \(y\le z\) für \(x,y,z\in\mathbb{R}\), so ist \(x \le z\).

\item {} 
Ist \(x \le y\) und \(y \le x\) für \(x,y \in \mathbb{R}\), so ist \(x = y\).

\end{itemize}

(Mit \(x<y\) bezeichnen wir den Fall \(x \le y\) und \(x \not= y\).)

\item {} 
Für je zwei Elemente \(x,y \in\mathbb{R}\) gilt
\begin{equation*}
\begin{split}x \le y\quad \text{oder}\quad y\le x.\end{split}
\end{equation*}
\end{enumerate}

\item {} 
Die Ordnung ist verträglich mit Addition und Multiplikation, d.h. für \(a,b,c\in\mathbb{R}\) gilt

\end{itemize}
\begin{equation*}
\begin{split}\begin{array}{c}
  a \le b \Rightarrow a+c \le b+c\\
  a \le b, 0 \le c \Rightarrow a\cdot c \le b\cdot c
  \end{array}\end{split}
\end{equation*}
\item {} 
\(\mathbb{R}\) ist vollständig. Das heisst, dass jede nicht leere nach oben beschränkte Menge reeller Zahlen eine kleinste obere Schranke besitzt.

\end{itemize}
\end{sphinxadmonition}

Im Allgemeinen ist ein (Zahlen)\sphinxhyphen{}Körper über eine Gruppe wie folgt definiert:

\begin{sphinxadmonition}{note}{Definition: Gruppe}

Ein Tupel \((G, \cdot)\) bestehend aus einer Menge \(G\) und einer Verknüpfung \(\cdot : G \to G\) heisst \sphinxstyleemphasis{Gruppe}, falls die Verknüpfung assoziativ ist, ein neutrales Element \(e\in G\) existiert und für alle \(a \in G\) ein \(b\in G\) exisitert so dass \(a\cdot b = b\cdot a = e\) gilt. Ist die Verknüpfung kommutativ, nennt man die Gruppe kommutativ.
\end{sphinxadmonition}

Ein Körper lässt sich somit wie folgt allgemein definieren:

\begin{sphinxadmonition}{note}{Definition: Körper}

Ein \sphinxstyleemphasis{Körper} ist eine Tripel \((K, +, \cdot)\) mit folgenden Eigenschaften:
\begin{itemize}
\item {} 
\((K, +)\) ist eine kommutative Gruppe mit neutralem Element \(0_K\).

\item {} 
\((K, \cdot)\) ist eine kommutative Gruppe mit neutralem Element \(1_K\).

\item {} 
(Distributivgesetz) Für alle \(a,b,c\in K\) gilt

\end{itemize}
\begin{equation*}
\begin{split}\begin{split}a\cdot (b+c) & = a\cdot b + a\cdot c\\
(a+b)\cdot c & = (a\cdot c) + (b\cdot c)\end{split}\end{split}
\end{equation*}\end{sphinxadmonition}

Beispiele für Körper sind folgende Tupel
\begin{itemize}
\item {} 
\((\mathbb{R}, +, \cdot)\) und \((\mathbb{Q}, +, \cdot)\)  mit der üblichen Addition und Multiplikation,

\item {} 
\((\mathbb{Z}, +, \cdot)\) bildet \sphinxstylestrong{keine} Gruppe.

\end{itemize}

Mit Hilfe eines Körpers können wir nun einen \(\mathbb{K}\)\sphinxhyphen{}Vektorraum wie folgt definieren:

\begin{sphinxadmonition}{note}{Definition: \protect\(\mathbb{K}-Vektorraum\protect\)}

Ein \(\mathbb{K}\)\sphinxhyphen{}Vektorraum ist ein Tripel \((V, +, \cdot)\) mit den Eigenschaften
\begin{enumerate}
\sphinxsetlistlabels{\arabic}{enumi}{enumii}{}{.}%
\item {} 
\((V,+)\) ist eine kommutative Gruppe

\item {} 
Die Abbildung \(\cdot : \mathbb{K} \times V \to V\) genügt den Eigenschaften
\begin{itemize}
\item {} 
\(\alpha\cdot (\beta\cdot x) = (\alpha\cdot \beta)\cdot x\)

\item {} 
\(\alpha\cdot (x+y) = (\alpha\cdot x) + (\alpha\cdot y)\)

\item {} 
\((\alpha+\beta)\cdot x = (\alpha\cdot x) + (\beta\cdot x)\)

\item {} 
\(1_K \cdot x = x\)

\end{itemize}

wobei \(\alpha, \beta\in\mathbb{K},\ x, y\in V\).

\end{enumerate}
\end{sphinxadmonition}

\begin{sphinxadmonition}{note}{Bemerkung}

Wenn man in einem allgemeinen Kontext von Vektoren spricht, so meint man damit die Elemente eines Vektorraumes. Spricht man von Skalaren, so sind die Elemente des zugrundeliegenden Körpers gemeint.
\end{sphinxadmonition}

\sphinxstylestrong{Beispiele}:
\begin{itemize}
\item {} 
Vektorraum \(\mathbb{R}^n\)

Anwendung in python:

\end{itemize}

\begin{sphinxVerbatim}[commandchars=\\\{\}]
\PYG{k+kn}{import} \PYG{n+nn}{numpy} \PYG{k}{as} \PYG{n+nn}{np}
\PYG{n}{x} \PYG{o}{=} \PYG{n}{np}\PYG{o}{.}\PYG{n}{array}\PYG{p}{(}\PYG{p}{[}\PYG{l+m+mi}{1}\PYG{p}{,}\PYG{l+m+mi}{2}\PYG{p}{,}\PYG{l+m+mi}{3}\PYG{p}{,}\PYG{l+m+mi}{4}\PYG{p}{]}\PYG{p}{)}
\PYG{n}{y} \PYG{o}{=} \PYG{n}{np}\PYG{o}{.}\PYG{n}{array}\PYG{p}{(}\PYG{p}{[}\PYG{o}{\PYGZhy{}}\PYG{l+m+mi}{4}\PYG{p}{,}\PYG{o}{\PYGZhy{}}\PYG{l+m+mi}{3}\PYG{p}{,}\PYG{o}{\PYGZhy{}}\PYG{l+m+mi}{2}\PYG{p}{,}\PYG{o}{\PYGZhy{}}\PYG{l+m+mi}{1}\PYG{p}{]}\PYG{p}{)}

\PYG{n+nb}{print}\PYG{p}{(}\PYG{l+s+s1}{\PYGZsq{}}\PYG{l+s+s1}{x+y=}\PYG{l+s+s1}{\PYGZsq{}}\PYG{p}{,}\PYG{n}{x}\PYG{o}{+}\PYG{n}{y}\PYG{p}{)}
\PYG{n+nb}{print}\PYG{p}{(}\PYG{l+s+s1}{\PYGZsq{}}\PYG{l+s+s1}{5*(x+y)=}\PYG{l+s+s1}{\PYGZsq{}}\PYG{p}{,}\PYG{l+m+mi}{5}\PYG{o}{*}\PYG{p}{(}\PYG{n}{x}\PYG{o}{+}\PYG{n}{y}\PYG{p}{)}\PYG{p}{,}\PYG{l+s+s1}{\PYGZsq{}}\PYG{l+s+s1}{= 5*x+5*y = }\PYG{l+s+s1}{\PYGZsq{}}\PYG{p}{,}\PYG{l+m+mi}{5}\PYG{o}{*}\PYG{n}{x}\PYG{o}{+}\PYG{l+m+mi}{5}\PYG{o}{*}\PYG{n}{y}\PYG{p}{)}
\end{sphinxVerbatim}

\begin{sphinxVerbatim}[commandchars=\\\{\}]
x+y= [\PYGZhy{}3 \PYGZhy{}1  1  3]
5*(x+y)= [\PYGZhy{}15  \PYGZhy{}5   5  15] = 5*x+5*y =  [\PYGZhy{}15  \PYGZhy{}5   5  15]
\end{sphinxVerbatim}
\begin{itemize}
\item {} 
Vektorraum der stetigen Funktionen.

Sei \(\alpha \in \mathbb{R}\) und \(u,v : [a,b]\subset \mathbb{R} \to \mathbb{R}\) zwei stetige Funktionen. Zeige, dass die Summe zweier stetiger Funktionen \((u+v)\) wieder stetig ist und dass das Vielfache einer stetigen Funktion \((\alpha u)\) ebenso stetig ist.

\begin{sphinxadmonition}{note}{Aufgabe}

Beweise die Aussage.
\end{sphinxadmonition}

\end{itemize}


\subsection{Metrische Räume}
\label{\detokenize{Funktionalanalysis/Funktionalanalysis:metrische-raume}}
In der Analysis will man oft eine Distanz zwischen zwei Elemente eines Vektorraumes angeben. Insbesondere bei Konvergenzbetrachtungen ist der Abstand zweier Elemente existentiell wichtig. Der Konvergenzbegriff in \(\mathbb{R}\) unter Hinzunahme der folgenden Distanzfunktion
\begin{equation*}
\begin{split}\begin{split} d : \mathbb{R}\times\mathbb{R} & \to \mathbb{R}^+\\
(x,y) & \mapsto  d = d(x,y) := |x-y|\end{split}\end{split}
\end{equation*}
lautet wie folgt: Die Folge \(\{x_n\}\) aus \(\mathbb{R}\) heisst konvergent gegen \(x_0\in\mathbb{R}\), wenn es zu jedem \(\varepsilon > 0\) eine natürliche Zahl \(n_0 = n_0(\varepsilon)\) gibt, so dass
\begin{equation*}
\begin{split}d(x_n,x_0) = |x_n-x_0| < \varepsilon\end{split}
\end{equation*}
für alle \(n \ge n_0\) gilt.

\begin{sphinxadmonition}{note}{Definition: Metrik}

Eine nichtleere Menge \(X\) mit \sphinxstyleemphasis{Elemente} \(x, y, z, \ldots\) heisst ein \sphinxstyleemphasis{metrischer Raum}, wenn jedem Paar \(x, y \in X\) eine reelle Zahl \(d(x,y)\), genannt \sphinxstyleemphasis{Abstand} oder \sphinxstyleemphasis{Metrik}, zugeordnet ist, mit den Eigenschaften: Für alle \(x,y,z\in X\) gilt
\begin{enumerate}
\sphinxsetlistlabels{\arabic}{enumi}{enumii}{}{.}%
\item {} 
\(d(x,y) \ge 0,\ d(x,y) = 0\) genau dann, wenn \(x=y\) ist

\item {} 
\(d(x,y) = d(y,x)\) \sphinxstyleemphasis{Symmetrieeigenschaft}

\item {} 
\(d(x,y) \le d(x,z) + d(z,y)\) \sphinxstyleemphasis{Dreiecksungleichung}.

\end{enumerate}
\end{sphinxadmonition}

Für metrische Räume verwenden wir wieder die Schreibweisen: \((X, d)\) oder kurz \(X\), falls der Kontext klar ist.

\begin{sphinxadmonition}{note}{Aufgabe}

Zeige, dass für \(X = C[0,1]\), den stetigen Funktionen auf dem Intervall \([0,1]\) die Abbildung
\begin{equation*}
\begin{split}\begin{split}d_{\text{max}} : X \times X & \to \mathbb{R}^+\\
(x,y) & \mapsto d_{\text{max}}(x,y) := \max_{t\in[0,1]} |x(t)-y(t)|\end{split}\end{split}
\end{equation*}
eine Metrik definiert (\sphinxstyleemphasis{Maximumsmetrik})
\end{sphinxadmonition}

\sphinxstylestrong{Beispiel}:

\begin{sphinxVerbatim}[commandchars=\\\{\}]
\PYG{k+kn}{import} \PYG{n+nn}{numpy} \PYG{k}{as} \PYG{n+nn}{np}
\PYG{k+kn}{import} \PYG{n+nn}{matplotlib}\PYG{n+nn}{.}\PYG{n+nn}{pyplot} \PYG{k}{as} \PYG{n+nn}{plt}

\PYG{n}{x} \PYG{o}{=} \PYG{k}{lambda} \PYG{n}{t}\PYG{p}{:} \PYG{n}{t}\PYG{o}{*}\PYG{o}{*}\PYG{l+m+mi}{2}
\PYG{n}{y} \PYG{o}{=} \PYG{k}{lambda} \PYG{n}{t}\PYG{p}{:} \PYG{n}{t}\PYG{o}{*}\PYG{p}{(}\PYG{l+m+mi}{1}\PYG{o}{\PYGZhy{}}\PYG{n}{t}\PYG{p}{)}\PYG{o}{+}\PYG{l+m+mi}{2}\PYG{o}{/}\PYG{p}{(}\PYG{l+m+mi}{1}\PYG{o}{+}\PYG{p}{(}\PYG{p}{(}\PYG{n}{t}\PYG{o}{\PYGZhy{}}\PYG{l+m+mf}{0.5}\PYG{p}{)}\PYG{o}{/}\PYG{l+m+mf}{0.02}\PYG{p}{)}\PYG{o}{*}\PYG{o}{*}\PYG{l+m+mi}{2}\PYG{p}{)}

\PYG{n}{t} \PYG{o}{=} \PYG{n}{np}\PYG{o}{.}\PYG{n}{linspace}\PYG{p}{(}\PYG{l+m+mi}{0}\PYG{p}{,}\PYG{l+m+mi}{1}\PYG{p}{,}\PYG{l+m+mi}{400}\PYG{p}{)}

\PYG{n}{plt}\PYG{o}{.}\PYG{n}{plot}\PYG{p}{(}\PYG{n}{t}\PYG{p}{,}\PYG{n}{x}\PYG{p}{(}\PYG{n}{t}\PYG{p}{)}\PYG{p}{,} \PYG{n}{label}\PYG{o}{=}\PYG{l+s+s1}{\PYGZsq{}}\PYG{l+s+s1}{\PYGZdl{}x(t)\PYGZdl{}}\PYG{l+s+s1}{\PYGZsq{}}\PYG{p}{)}
\PYG{n}{plt}\PYG{o}{.}\PYG{n}{plot}\PYG{p}{(}\PYG{n}{t}\PYG{p}{,}\PYG{n}{y}\PYG{p}{(}\PYG{n}{t}\PYG{p}{)}\PYG{p}{,} \PYG{n}{label}\PYG{o}{=}\PYG{l+s+s1}{\PYGZsq{}}\PYG{l+s+s1}{\PYGZdl{}y(t)\PYGZdl{}}\PYG{l+s+s1}{\PYGZsq{}}\PYG{p}{)}
\PYG{n}{plt}\PYG{o}{.}\PYG{n}{legend}\PYG{p}{(}\PYG{p}{)}
\PYG{n}{plt}\PYG{o}{.}\PYG{n}{show}\PYG{p}{(}\PYG{p}{)}
\end{sphinxVerbatim}

\noindent\sphinxincludegraphics{{Funktionalanalysis_3_0}.png}

Die Maximummetrik berechnet die maximale Differenz der beiden Funktionen:

\begin{sphinxVerbatim}[commandchars=\\\{\}]
\PYG{n}{plt}\PYG{o}{.}\PYG{n}{plot}\PYG{p}{(}\PYG{n}{t}\PYG{p}{,}\PYG{n}{np}\PYG{o}{.}\PYG{n}{abs}\PYG{p}{(}\PYG{n}{x}\PYG{p}{(}\PYG{n}{t}\PYG{p}{)}\PYG{o}{\PYGZhy{}}\PYG{n}{y}\PYG{p}{(}\PYG{n}{t}\PYG{p}{)}\PYG{p}{)}\PYG{p}{,} \PYG{n}{label}\PYG{o}{=}\PYG{l+s+s1}{\PYGZsq{}}\PYG{l+s+s1}{\PYGZdl{}|x(t)\PYGZhy{}y(t)|\PYGZdl{}}\PYG{l+s+s1}{\PYGZsq{}}\PYG{p}{)}
\PYG{n}{plt}\PYG{o}{.}\PYG{n}{legend}\PYG{p}{(}\PYG{p}{)}
\PYG{n}{plt}\PYG{o}{.}\PYG{n}{show}\PYG{p}{(}\PYG{p}{)}
\end{sphinxVerbatim}

\noindent\sphinxincludegraphics{{Funktionalanalysis_5_0}.png}

\begin{sphinxadmonition}{note}{Aufgabe}

Berechne analytisch den exakten Wert der Maximumsmetrik für die beiden Funktionen \(x,y\) aus obigem Beispiel.
\end{sphinxadmonition}

Eine weitere wichtige Metrik ist die \sphinxstylestrong{Integralmetrik}. Es sei \(X\) die Menge aller reellwertigen Funktionen, die auf einem (nicht notwendig beschränkten) Intervall \((a,b)\) stetig sind und für die das Integral
\begin{equation*}
\begin{split}\int_a^b |x(t)|^p dt, \quad 1\le p < \infty\end{split}
\end{equation*}
im Riemannschen Sinne existiert. Setzen wir für \(x(t), y(t)\in X\)

\begin{sphinxadmonition}{note}{Definition: Integralmetrik}
\begin{equation*}
\begin{split}d_p(x,y) := \left(\int_a^b |x(t)-y(t)|^p dt\right)^{1/p},\quad 1\le p < \infty\end{split}
\end{equation*}\end{sphinxadmonition}

so wird \(X\) mit dieser \sphinxstyleemphasis{Integralmetrik} zu einem metrischen Raum. Der Beweis nutzt die Minkowski\sphinxhyphen{}Ungleichung für Integrale
\begin{equation*}
\begin{split}\left(\int_a^b |u(t)-v(t)|^p dt\right)^{1/p} \le \left(\int_a^b |u(t)|^p dt\right)^{1/p} + \left(\int_a^b |v(t)|^p dt\right)^{1/p}.\end{split}
\end{equation*}
\sphinxstylestrong{Beispiel}:

\begin{sphinxVerbatim}[commandchars=\\\{\}]
\PYG{n}{plt}\PYG{o}{.}\PYG{n}{plot}\PYG{p}{(}\PYG{n}{t}\PYG{p}{,}\PYG{n}{x}\PYG{p}{(}\PYG{n}{t}\PYG{p}{)}\PYG{p}{,} \PYG{n}{label}\PYG{o}{=}\PYG{l+s+s1}{\PYGZsq{}}\PYG{l+s+s1}{\PYGZdl{}x(t)\PYGZdl{}}\PYG{l+s+s1}{\PYGZsq{}}\PYG{p}{)}
\PYG{n}{plt}\PYG{o}{.}\PYG{n}{plot}\PYG{p}{(}\PYG{n}{t}\PYG{p}{,}\PYG{n}{y}\PYG{p}{(}\PYG{n}{t}\PYG{p}{)}\PYG{p}{,} \PYG{n}{label}\PYG{o}{=}\PYG{l+s+s1}{\PYGZsq{}}\PYG{l+s+s1}{\PYGZdl{}y(t)\PYGZdl{}}\PYG{l+s+s1}{\PYGZsq{}}\PYG{p}{)}
\PYG{n}{plt}\PYG{o}{.}\PYG{n}{fill\PYGZus{}between}\PYG{p}{(}\PYG{n}{t}\PYG{p}{,}\PYG{n}{x}\PYG{p}{(}\PYG{n}{t}\PYG{p}{)}\PYG{p}{,}\PYG{n}{y}\PYG{p}{(}\PYG{n}{t}\PYG{p}{)}\PYG{p}{,}\PYG{n}{alpha}\PYG{o}{=}\PYG{l+m+mf}{0.5}\PYG{p}{,} \PYG{n}{label}\PYG{o}{=}\PYG{l+s+s1}{\PYGZsq{}}\PYG{l+s+s1}{\PYGZdl{}x(t)\PYGZhy{}y(t)\PYGZdl{}}\PYG{l+s+s1}{\PYGZsq{}}\PYG{p}{)}
\PYG{n}{plt}\PYG{o}{.}\PYG{n}{legend}\PYG{p}{(}\PYG{p}{)}
\PYG{n}{plt}\PYG{o}{.}\PYG{n}{show}\PYG{p}{(}\PYG{p}{)}
\end{sphinxVerbatim}

\noindent\sphinxincludegraphics{{Funktionalanalysis_7_0}.png}

\begin{sphinxadmonition}{note}{Aufgabe}

Berechne die Integralmetrik für \(p=2\) und \(p=1/2\) für das Beispiel oben.
\end{sphinxadmonition}

\begin{sphinxadmonition}{note}{Aufgabe}

In der Codierungstheorie ist ein \(n\)\sphinxhyphen{}stelliges Binärwort ein \(n\)\sphinxhyphen{}Tuppel \((\xi_1, \ldots, \xi_n)\), wobei \(\xi_k \in \{0,1\}\) für \(k=1,\ldots, n\). Bezeichne \(X\) die Menge aller dieser Binärwörter. Für \(x = \xi_1 \xi_2 \ldots \xi_n\), \(y = \eta_1 \eta_2 \ldots \eta_n\) ist die \sphinxstyleemphasis{Hamming\sphinxhyphen{}Distanz} zwischen \(x\) und \(y\) durch
\begin{equation*}
\begin{split}d_H(x,y) := \text{Anzahl der Stellen an denen sich $x$ und $y$ unterscheiden}\end{split}
\end{equation*}
definiert. Zeige:
\begin{enumerate}
\sphinxsetlistlabels{\arabic}{enumi}{enumii}{}{.}%
\item {} 
\(d_H(x,y)\) lässt sich durch
\begin{equation*}
\begin{split}d_H(x,y) := \sum_{k=1}^n [(\xi_k+\eta_k) \mod 2]\end{split}
\end{equation*}
darstellen.

\item {} 
\((X,d_H)\) ist ein metrischer Raum.

\end{enumerate}
\end{sphinxadmonition}

Die topologischen Begriffe wie \sphinxstyleemphasis{offene Kugel}, \sphinxstyleemphasis{innerer Punkt}, \sphinxstyleemphasis{Häufungspunkt}, \sphinxstyleemphasis{abgeschlossen}, \sphinxstyleemphasis{beschränkt} lassen sich mit Hilfe der Metrix \(d\) für einen metrischen Raum \((X,d)\) direkt aus dem aus der Analysis bekannten \(\mathbb{R}^n\) übertragen.

\begin{sphinxadmonition}{note}{Definition: Konvergenz}

Eine Folge \(\{x_n\}\subset X\) von Elemente aus \(X\) heisst \sphinxstyleemphasis{konvergent}, wenn es ein \(x_0\in X\) gibt mit
\begin{equation*}
\begin{split}d(x_n,x_0) \to 0\quad \text{für}\quad n\to\infty,\end{split}
\end{equation*}
dh. wenn es zu jedem \(\varepsilon > 0\) ein \(n_0 = n(\varepsilon)\in\mathbb{N}\) gibt, mit
\begin{equation*}
\begin{split}d(x_n,x_0) < \varepsilon\quad\text{für alle}\quad n\ge n_0.\end{split}
\end{equation*}
Schreibweise: \(x_n \to x_0\) für \(n \to \infty\) oder \(\lim_{n\to\infty} x_n = x_0\), \(x_0\) heisst \sphinxstyleemphasis{Grenzwert} der Folge \(\{x_n\}\)
\end{sphinxadmonition}

Der Grenzwert einer konvergenten Folge ist eindeutig bestimmt. Dies lässt sich per Widerspruch wie folgt leicht zeigen: Seien \(x_0\) und \(y_0\) zwei verschiedene Grenzwerte. Daher gilt
\begin{equation*}
\begin{split}\begin{split}
0 < d(x_0,y_0) & \le d(x_0, x_n) + d(x_n,y_0)\\
& = \underbrace{d(x_n,x_0)}_{\to 0\ \text{für}\, n\to 0} + \underbrace{d(x_n,y_0)}_{\to 0\ \text{für}\, n\to 0}\\
& \to 0\ \text{für}\, n\to 0\end{split}.\end{split}
\end{equation*}
Es folgt damit \(d(x_0,y_0)=0\) und damit \(x_0 = y_0\).


\subsection{Normierte Räume. Banachräume}
\label{\detokenize{Funktionalanalysis/Funktionalanalysis:normierte-raume-banachraume}}

\subsection{Skalarprodukträume. Hilberträume}
\label{\detokenize{Funktionalanalysis/Funktionalanalysis:skalarproduktraume-hilbertraume}}

\section{Lineare Operationen}
\label{\detokenize{Funktionalanalysis/Funktionalanalysis:lineare-operationen}}

\section{Beschränkte lineare Operatoren}
\label{\detokenize{Funktionalanalysis/Funktionalanalysis:beschrankte-lineare-operatoren}}

\chapter{Beispiele zu normierte Räume}
\label{\detokenize{Funktionalanalysis/Beispiele:beispiele-zu-normierte-raume}}\label{\detokenize{Funktionalanalysis/Beispiele::doc}}

\section{Vektorraum \protect\(\mathbb{R}^n\protect\)}
\label{\detokenize{Funktionalanalysis/Beispiele:vektorraum-mathbb-r-n}}
\begin{sphinxVerbatim}[commandchars=\\\{\}]
\PYG{k+kn}{import} \PYG{n+nn}{numpy} \PYG{k}{as} \PYG{n+nn}{np}
\end{sphinxVerbatim}

\begin{sphinxVerbatim}[commandchars=\\\{\}]
\PYG{n}{x} \PYG{o}{=} \PYG{n}{np}\PYG{o}{.}\PYG{n}{array}\PYG{p}{(}\PYG{p}{[}\PYG{l+m+mi}{1}\PYG{p}{,}\PYG{l+m+mi}{2}\PYG{p}{,}\PYG{l+m+mi}{3}\PYG{p}{]}\PYG{p}{)}
\PYG{n}{y} \PYG{o}{=} \PYG{n}{np}\PYG{o}{.}\PYG{n}{array}\PYG{p}{(}\PYG{p}{[}\PYG{n}{k}\PYG{o}{*}\PYG{o}{*}\PYG{l+m+mi}{2} \PYG{k}{for} \PYG{n}{k} \PYG{o+ow}{in} \PYG{n+nb}{range}\PYG{p}{(}\PYG{l+m+mi}{1}\PYG{p}{,}\PYG{l+m+mi}{4}\PYG{p}{)}\PYG{p}{]}\PYG{p}{)}
\end{sphinxVerbatim}

Die Addition, Subtraktion zweier Vektoren liefert wieder ein Vektor:

\begin{sphinxVerbatim}[commandchars=\\\{\}]
\PYG{n}{x}\PYG{o}{+}\PYG{n}{y}
\end{sphinxVerbatim}

\begin{sphinxVerbatim}[commandchars=\\\{\}]
array([ 2,  6, 12])
\end{sphinxVerbatim}

\begin{sphinxVerbatim}[commandchars=\\\{\}]
\PYG{n}{x}
\end{sphinxVerbatim}

\begin{sphinxVerbatim}[commandchars=\\\{\}]
array([1, 2, 3])
\end{sphinxVerbatim}


\part{Teil 2}


\chapter{Einführung PDE}
\label{\detokenize{PDE/IntroPDE:einfuhrung-pde}}\label{\detokenize{PDE/IntroPDE::doc}}
Einführung in partielle Differentialgleichungen.


\part{Teil 3}


\chapter{Numerik für partielle Differentialgleichungen}
\label{\detokenize{NumerikPDE/NumerikPDE:numerik-fur-partielle-differentialgleichungen}}\label{\detokenize{NumerikPDE/NumerikPDE::doc}}

\part{Teil 4}


\chapter{Anwendungen}
\label{\detokenize{Applications/Applications:anwendungen}}\label{\detokenize{Applications/Applications::doc}}
Anwendungen







\renewcommand{\indexname}{Index}
\printindex
\end{document}